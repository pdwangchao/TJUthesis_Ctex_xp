% !Mode:: "TeX:UTF-8"

\chapter{特大空间声场的声XXXX}

本章旨在XXX。

\section{声能空XXX}

本节研究XXX。

\subsection{实测XXX}

天津党校报告XXX

 其平面图如图~\ref{fig:chapt_2_plan_HIT}所示,场地概况和测试环境描述同第\ref{sec:test_describe_and_measurements} 节。

a)测试设备:测试中所用声源为无指向球形声源声望OS002,位于比赛场地中央略偏$\rm 1m$位置,离地高度为$\rm 1.5m$。

\subsection{计算XX}

XXXX

其他未说明的模拟设置与章节\ref{sec:simulation_discribe}相同。


\subsection{模拟结果分析}

XXX。

XXXX
\begin{equation}
\label{classical_equation}
L^c=10log_{10}(\frac{W}{4{\pi}r^2}+\frac{4W}{R})+120
\end{equation}

单位为$\rm dB$。其中,$\rm L^c$为室内声压级经典计算公式的计算结果;$\rm W$为声源声功率(W);$\rm R$为房间常数;$\rm r$为距声源距离($\rm m$)。
XXX

\begin{table}[!htb]
\centering
\caption{不同容积下的降噪曲线线性回归}
\label{tab:chapt_3_regression_noise_reduction}
\vspace{0.5em}\centering\wuhao
\begin{tabular}{cccccccc}
\toprule[1.5pt]
\multirow{2}{*}{空间类型}                                               & \multirow{2}{*}{回归结果} & \multicolumn{6}{c}{空间宽度(m)}                         \\
                                                                    &                                   & 20     & 40     & 60     & 80     & 100    & 150    \\ \midrule[1pt]
\multirow{2}{*}{类型~\uppercase\expandafter{\romannumeral1}} & 斜率                             & -10.02 & -9.93  & -9.53  & -9.11  & -8.73  & -7.95  \\
                                                                    & $R^2$                                & 0.954  & 0.971  & 0.98   & 0.986  & 0.989  & 0.995  \\
\multirow{2}{*}{类型~\uppercase\expandafter{\romannumeral2}} & 斜率                             & -14.6  & -14.06 & -13.46 & -12.93 & -12.46 & -11.49 \\
                                                                    & $R^2$                                & 0.974  & 0.982  & 0.986  & 0.99   & 0.992  & 0.995  \\
\bottomrule[1.5pt]
\end{tabular}
\vspace{\baselineskip}
\end{table}

\subsection{不均匀XXX}


\FloatBarrier
\section{本章小结}
本章的研究目XXX。
